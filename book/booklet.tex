\documentclass[paper=a4, twoside=false, fontsize=12pt]{scrbook}

% General packages
\usepackage{sourdough}
\usepackage[
  paperwidth=210mm,
  paperheight=260mm,
  top=10mm,
  bottom=80mm,
  inner=10mm,
  outer=10mm,
  marginparsep=7mm,
  marginparwidth=48mm,
]{geometry}
\usepackage{subcaption}

\pagenumbering{gobble}
% Basic attributes
\author{Hendrik Kleinwächter}
\title{The Sourdough Framework\\\texttt{tl;dr Booklet Version}}
\begin{document}
\maketitle

\section*{Sourdough starter}
\begin{flowchart}[!htb]
    \centering
    \input{figures/fig-starter-process.tex}
    \caption*{How to setup a sourdough starter}
\end{flowchart}

\begin{flowchart}[!htb]
    \centering
    \begin{tikzpicture}[node distance = 3cm, auto]
  \node [start] (init) {Create a starter};
  \node [decision, right of=init, node distance=3.5cm] (decision_start) {Starter last fed within 3~days?};
  \node [block, right of=decision_start, text width=7em, node distance=4cm] (feed_no_branch)
      {Feed starter twice:\par \qty{48}{\hour} before\par \qtyrange{6}{12}{\hour} before};
  \node [block, below of=feed_no_branch, text width=7em, node distance=2.7cm] (feed_yes_branch)
      {Feed starter \qtyrange{6}{12}{\hour} before making dough.};
  \node [block, right of=feed_no_branch, text width=7em, node distance=4cm] (high_ratio)
    {Use a 1:10:10 ratio:\par \begin{tabular}{r@{}l}
           10&~g starter,\\
          100&~g flour, \\
          100&~g water.\end{tabular}};
  \node [block, right of=feed_yes_branch, text width=7em, node distance=4cm] (low_ratio)
  {Use a 1:5:5 ratio:\par \begin{tabular}{r@{}l}
          10&~g starter,\\
          50&~g flour, \\
          50&~g water.\end{tabular}};
  \node [decision, below of=high_ratio, node distance=6cm] (size_check)
    {Bubbly? Increased in size?};
  \node [decision, below of=decision_start, node distance=6cm] (smell_check)
    {Vinegary or yogurty smell?};
  \node [success, below of=init, node distance=6cm] (make_dough)
    {Prepare dough};

  \path [line] (init) -- (decision_start);
  \path [line] (decision_start) -- node{no} (feed_no_branch);
  \path [line] (decision_start) -- node[below=2pt]{yes} (feed_yes_branch.north west);
  \path [line] (feed_yes_branch) -- (low_ratio);
  \path [line] (feed_no_branch) -- (high_ratio);
  \path [line] (high_ratio) -- node[anchor=east, above=2pt] {} ++(2.2,0) |-(size_check);
  \path [line] (low_ratio) -- (size_check);
  \path [line] (size_check) -- node{no} (feed_yes_branch.south east);
  \path [line] (size_check) -- node{yes} (smell_check);
  \path [line] (smell_check) -- node{no} (feed_yes_branch.south west);
  \path [line] (smell_check) -- node{yes} (make_dough);

  % braces
  \draw[BC]   (size_check.south) -- 
      node[below=1em]{Check if starter is ready to be used}(smell_check.south);
\end{tikzpicture}

    \caption*{Preparing your starter for baking}
\end{flowchart}

\begin{flowchart}[!htb]
    \centering
    \input{figures/fig-starter-maintenance.tex}
    \caption*{Maintaining your starter, change ratio as per starter hydration
    type}
\end{flowchart}

\clearpage{}
\section*{Baker's math}
\begin{table}[!htb]
  \centering
  \input{tables/table-bakers-math-example.tex}
  \caption*{An example table demonstrating how to properly calculate using
  baker's math. All the ingredients are calculated as a percentage of the
flour quantity.}
\end{table}

\section*{Basic recipes}
\subsection*{Flat bread}
\subsubsection*{Ingredients}
\begin{tabular}{r@{}rl@{}}
\qty{400}{\gram} &~(\qty{100}{\percent}) & Flour (wheat, rye, corn, whatever
                                            you have at hand)\\
\qty{320}{\gram} &  (\qty{80}{\percent}) & Water, preferably at room
                                            temperature\\
\qty{80}{\gram}  &  (\qty{20}{\percent}) & Active sourdough starter\\
\qty{8}{\gram}   &   (\qty{2}{\percent}) & Salt\\
\end{tabular}

\subsubsection*{Instructions}
\begin{description}
\item[Prepare the dough] In a large mixing bowl, combine the flour and water.
    Mix until you have a shaggy dough with no dry spots.

    Add the sourdough starter and salt to the mixture. Incorporate them
    thoroughly until you achieve a smooth and homogenized dough.

\item[Fermentation:] Cover the bowl with a lid or plastic wrap. Allow the dough
    to rest and ferment until it has increased by at least \qty{50}{\percent}
    in size.  Depending on the temperature and activity of your starter, this
    can take anywhere from 4 to 24~hours.

\item[Cooking preparation:] Once the dough has risen, heat a pan over medium
    heat.  Lightly oil the pan, ensuring to wipe away any excess oil with a
    paper towel.

\item[Shaping and cooking:] With a ladle or your hands, scoop out a portion of
    the dough and place it onto the hot pan, spreading it gently like a
    pancake.

    Cover the pan with a lid. This traps the steam and ensures even cooking
    from the top, allowing for easier flipping later.

    After about 5~minutes, or when the bottom of the flatbread has a
    golden-brown crust, carefully flip it using a spatula.

    \emph{Adjusting cook time.} If the flatbread appears too dark, remember to
    reduce the cooking time slightly for the next one.  Conversely, if it's
    too pale, allow it to cook a bit longer before flipping.

    Cook the flipped side for an additional 5~minutes or until it's also
    golden brown.

\item[Storing:] Once cooked, remove the flatbread from the pan and place it on
    a kitchen towel. Wrapping the breads in the towel will help retain their
    softness and prevent them from becoming overly crisp.  Repeat the cooking
    process for the remaining dough.

\item[Serving suggestion:] Enjoy your sourdough flatbreads warm, paired with
    your favorite dips, spreads, or as a side to any meal.

\end{description}

\clearpage{}

\subsection*{Freestanding \& sandwich wheat-based breads}
\begin{table}[!htb]
\centering
    \begin{tabular}{@{}lrrrp{0.4\linewidth}@{}}
    \toprule
    \thead{Ingredient}&                   & \thead{Percentage}  & \thead{Calculation} & \thead{Comments} \\ \midrule
    Flour             & \qty{400}{g}      &                     &                     & \\ 
    Whole-wheat flour & \qty{100}{g}      &                     &                     & \\ 
    Total flour       &                   & \qty{100}{\percent} & \qty{500}{g}        & \\
    Water             &                   & \qty{60}{\percent}  & \qty{300}{g}        & \\
    Sourdough starter &                   & \qty{10}{\percent}  & \qty{50}{g}         & \\
    Salt              &                   & \qty{2}{\percent}   & \qty{10}{g}         & \\ \midrule
    Flour             &                   & \qty{100}{\percent} &                     & \\ 
    Water             & & & & \\
    Sourdough starter & & & & \\
    Salt              & & & & \\ \midrule
    Flour             & & & & \\ 
                      & & & & \\
                      & & & & \\
                      & & & & \\
                      & & & & \\ \bottomrule
    \end{tabular}
\caption*{Table for your own calculation using baker's math}
\end{table}

\begin{flowchart}[!htb]
    \centering
    \input{figures/fig-wheat-sourdough-process.tex}
    \caption*{The whole process of making wheat based sourdough breads}
\end{flowchart}

\begin{flowchart}[!htb]
    \centering
    \input{figures/fig-kneading-process.tex}
    \caption*{The kneading process to create dough strength}
\end{flowchart}

\begin{flowchart}[!htb]
    \centering
    \input{figures/fig-bulk-fermentation.tex}
    \caption*{How to properly manage bulk fermentation}
\end{flowchart}

\begin{figure*}[!htb]
  \centering
  \includegraphics[width=\textwidth]{stretch-and-fold-steps}
  \caption*{An overview of the steps involved to perform stretch and folds for
  wheat-based doughs. They are optional and should only be done when the dough
flattened out a lot.}%
\end{figure*}
\clearpage{}

\section*{Shaping}

\begin{figure*}[!htb]
\centering
    \begin{subfigure}{.475\linewidth}
      \includegraphics[width=\linewidth]{preshape-direction}
      \caption*{Preshaping: Drag the dough in the direction of the rough
      surface area.}%
    \end{subfigure}
    \begin{subfigure}{.475\linewidth}
      \includegraphics[width=\linewidth]{step-1-flour-applied}
      \caption*{Step 1: Apply flour to the dough's surface.}%
    \end{subfigure}\hfill % <-- "\hfill"
    \medskip % create some *vertical* separation between the graphs
    \begin{subfigure}{.475\linewidth}
      \includegraphics[width=\linewidth]{step-2-flipped-over}
      \caption*{Step 2: Flipp-over dough. Note how the sticky side is facing
      you while the floured side is facing the countertop.}
    \end{subfigure}\hfill % <-- "\hfill"
    \begin{subfigure}{.475\linewidth}
      \includegraphics[width=\linewidth]{step-3-rectangular}
      \caption*{Step 3: Make the dough rectangular, keep the sticky side
      facing you while the floured side is facing the countertop.}%
    \end{subfigure}
    \caption*{First steps of shaping process}
\end{figure*}

\begin{figure*}[htb!]
  \centering
  \includegraphics[width=\textwidth]{step-4-folding}
  \caption*{Step 4: The process of folding a batard.  Note how the rectangle
  is first glued together and then rolled inwards to create a dough roll.
  Ultimately the edges are sealed to create a more uniform dough.}%
\end{figure*}
\clearpage{}

\section*{Proofing}
\begin{flowchart}[!htb]
  \centering
  \input{figures/fig-proofing-process.tex}
\end{flowchart}
\clearpage{}

\section*{Baking}
\begin{flowchart}[!htb]
    \centering
    \input{figures/fig-baking-process.tex}
    \caption*{Summary of different bread baking processes}
\end{flowchart}


\begin{flowchart}[!htb]
    \centering
    \input{figures/fig-inverted-tray-method.tex}
    \caption*{Baking with the inverted tray method}
\end{flowchart}

\begin{flowchart*}[!htb]
    \centering
    \input{figures/fig-dutch-oven-process.tex}
    \caption*{Baking with a Dutch Oven}
\end{flowchart*}
\clearpage{}
\end{document}
